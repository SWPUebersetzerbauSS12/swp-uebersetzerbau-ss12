\documentclass[ngerman]{scrartcl}
\usepackage[T1]{fontenc}
\usepackage[latin9]{inputenc}
\usepackage{fullpage}
\usepackage{babel}

\begin{document}
\title{Maschinencodegenerierung}
\subtitle{SWP �bersetzerbau SS12 (Gruppe F)}
\maketitle

\section*{Einleitung}

Ziel unseres Teilprojektes war es aus gegebenem, optimierten LLVM-Code
ausf�hrbaren Maschinencode zu erzeugen. Hierbei konzentrierten wir
uns zun�chst auf die Erstellung von unter Linux-Systemen funktionierendem
GNU-Assembler (32 Bit). Sp�ter erm�glichten wir auch noch die Ausgabe
als Intel-Assembler.


\section*{Programmablauf}

Der eingegebene LLVM-Code wird durch den Lexer (FileLexer) geparst.
Dieser speichert die relevanten Informationen in Token. Anschlie�end
werden die Token durch den Translator in den jeweils relevanten Maschinencode
�bersetzt.


\section*{Hauptkomponenten}


\subsection*{CodeGenerator}


\subsection*{Lexer}

Da es sich bei LLVM um Drei-Adress-Code handelt lassen sich die relevanten
Informationen der einzelnen LLVM-Befehle grob aufteilen in: Art des
Befehls, Ziel und Operanden. Bei Methodendefinitionen werden zus�tzlich
die Parameter gespeichert.


\subsection*{Translator}

Der Translator greift auf eine Umsetzung der abstrakten Klasse Assembler
zu, z.B. GNUAssembler. Mit den dort zur Verf�gung stehenden Funktionen
wird zun�chst die Grundstruktur f�r ein ausf�hrbares Programm in der
jeweiligen Architektur erstellt. Dann werden die vom Lexer erstellten
Token durchlaufen, �bersetzt und in das Grundger�st eingef�gt. F�r
die �bersetzung von Variablennamen zu Speicheradressen greift der
Translator auf die Funktionen der Speicherverwaltung (MemoryManager)
zu.


\subsection*{MemoryManager}

Der MemoryManager verwaltet den f�r ein Programm zur Verf�gung stehenden
Speicher. In den einfachsten F�llen gibt er also bei Anfrage freie
Speicherbereiche in Form von Registern oder Stack-Adressen zur�ck
und ordnet bereits existierenden Variablen ihre aktuellen Adressen
zu. Nat�rlich muss dies je nach Art der Variable unterschiedlich behandelt
werden.

Hierbei wird zwischen dem ,,globalen`` Speicher und den einzelnen
LLVM-Namespaces, also den Zuordnungen innerhalb der einzelnen Methoden
(MemoryContext) unterschieden.

Dar�ber hinaus gibt es noch die M�glichkeit Variablen von Registern
auf den Stack zu verschieben, um sie z. B. vor einem Funktionsaufruf
zu sichern.


\section*{Probleme}

Da wir f�r die meiste Zeit des Projektes keinen LLVM-Code zur Verf�gung
hatten, der innerhalb des Projektes aus der Quellsprache erzeugt wurde,
haben wir uns haupts�chlich an Code orientiert, der mit Standardtools
(clang) aus C-Code erzeugt wurde.

Nat�rlich gab es beim intern erzeugten LLVM-Code Unterschiede, auf
die wir dann kurzfristig reagieren mussten, dies war besonders bei
der Umsetzung von Strings problematisch.

Da wir also kaum intern erzeugten Code zur Verf�gung hatten, waren
wir auch beim Testen und Debuggen auf externen Code angewiesen, daher
konnten wir Tests als Teil der gesamten Compilerkette nicht in dem
Ma�e ausf�hren, wie wir es gerne getan h�tten.
\end{document}
