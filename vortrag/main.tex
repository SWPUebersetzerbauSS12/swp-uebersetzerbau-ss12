\documentclass[ucs,9pt]{beamer}
% frueher mal:
% \documentclass[hyperref={pdfpagelabels=false},ucs,9pt]{beamer}

% Copyright 2004 by Till Tantau <tantau@users.sourceforge.net>.
%
% In principle, this file can be redistributed and/or modified under
% the terms of the GNU Public License, version 2.
%
% However, this file is supposed to be a template to be modified
% for your own needs. For this reason, if you use this file as a
% template and not specifically distribute it as part of a another
% package/program, I grant the extra permission to freely copy and
% modify this file as you see fit and even to delete this copyright
% notice.
%
% Modified by Tobias G. Pfeiffer <tobias.pfeiffer@math.fu-berlin.de>
% to show usage of some features specific to the FU Berlin template.

% remove this line and the "ucs" option to the documentclass when your editor is not utf8-capable
\usepackage[utf8x]{inputenc}    % to make utf-8 input possible
\usepackage[ngerman]{babel}     % hyphenation etc., alternatively use 'german' as parameter
\usepackage{graphicx}
\usepackage{listings}
\usepackage{multirow}

\include{fu-beamer-template}  % THIS is the line that includes the FU template!
%\include{algorithmic}

\usepackage{arev,t1enc} % looks nicer than the standard sans-serif font
% if you experience problems, comment out the line above and change
% the documentclass option "9pt" to "10pt"

% image to be shown on the title page (without file extension, should be pdf or png)
\titleimage{fu_500}

\title[Softwareprojekt Übersetzerbau]
{Softwareprojekt Übersetzerbau}

\subtitle{Optimierungstechniken}

\author[Knötel, Karger, Marzin] % (optional, use only with lots of authors)
{David~Knötel, Björn~Karger, Daniel Marzin}
% - Give the names in the same order as the appear in the paper.

\institute[FU Berlin] % (optional, but mostly needed)
{Freie Universität Berlin, Institut für Informatik}
% - Keep it simple, no one is interested in your street address.

\date[20.07.2012] % (optional, should be abbreviation of conference name)
{}
% - Either use conference name or its abbreviation.
% - Not really informative to the audience, more for people (including
%   yourself) who are reading the slides online

%\subject{Verteidigung der Bachelorarbeit}
% This is only inserted into the PDF information catalog. Can be left
% out.

% you can redefine the text shown in the footline. use a combination of
% \insertshortauthor, \insertshortinstitute, \insertshorttitle, \insertshortdate, ...
\renewcommand{\footlinetext}{\insertshorttitle, \insertshortdate}

\setbeamertemplate{navigation symbols}{}

% Delete this, if you do not want the table of contents to pop up at
% the beginning of each subsection:
\AtBeginSection[] {
	\begin{frame}<beamer>{Leitfaden}
		\tableofcontents[currentsection,currentsubsection]
	\end{frame}
}

\begin{document}

\frame[plain]{\titlepage}
%\begin{frame}[plain]
%	\titlepage
%\end{frame}

\lstset{language=C,showstringspaces=false,tabsize=4,keywordstyle=\color{red},commentstyle=\color{blue},
stringstyle=\color{brown},frame=single}

% -------------------------------------------------------------------------------------------
% -------------------------------------------------------------------------------------------

\begin{frame}{Einführung}
\includegraphics[scale=1]{bilder/einfuehrung}
\end{frame}

\begin{frame}{Programmstruktur}
	\includegraphics[scale=0.4]{UML-Vortrag.png}
\end{frame}

\begin{frame}{Implementierung}
\includegraphics[scale=0.8]{bilder/ssa}
\end{frame}

\begin{frame}{Optimierungen \& Beispiele}

\begin{columns}
\begin{column}{.48\textwidth}

\begin{enumerate}
\item <+-| alert@+> Constant folding
\item <+-| alert@+> Constant propagation
\item <+-| alert@+> Reaching definition analysis
\item <+-| alert@+> Eliminate dead registers/blocks
\item <+-| alert@+> Remove local common subexpressions
\item <+-| alert@+> Global liveness analysis
\item <+-| alert@+> Strength reduction
\end{enumerate}

\end{column}%
%\hfill%
\begin{column}{.48\textwidth}
%\color{green}\rule{\linewidth}{2pt}

{\fbox{\parbox[c][2.9cm]{5cm}{
\only<1-1>{
{\color{blue} \%a = add i32 1, 7}}
\only<2-2>{
{\color{blue}\%a = add i32 8, 0\\
\%b = sub i32 \%a, 5}
}
\only<3-3>{
{\color{blue}
  store i32 5, i32* \%a, align 4\\
  \%2 = load i32* \%a, align 4
}}
\only<4-4>{
{\color{blue}br i1  1, label \%j1, label \%j2\\

j1: \%a = add i32 1, 7\\
j2: \%b = add i32 8, 0}
}
\only<5-5>{
{\color{blue}	\%q = load i32* \%x, align 4\\
\%w = load i32* \%x, align 4
}}
\only<6-6>{
{\color{blue}	\%a = alloca i32, align 4\\
	store i32 1, i32* \%a, align 4\\
	ret i32 0
}}
\only<7-7>{
{\color{blue}	\%a = mul i32 4, \%b
}}
}}\\
{\color{red}unoptimierter Code}\\
\vspace*{0.3cm}
\fbox{\parbox[c][2.9cm]{5cm}{
\only<1-1>{
{\color{blue}\%a = add i32 8, 0}\\
(bei uns eine Zuweisung)\\}
\only<2-2>{
{\color{blue}\%b = sub i32 8, 5}}
\only<3-3>{
{\color{blue}
  \%2 = add i32 5, 0
}}
\only<4-4>{
{\color{blue}br label \%j1\\

j1: \%a = add i32 1, 7
}}
\only<5-5>{
{\color{blue}	\%q = load i32* \%x, align 4\\
\%w = add i32 \%q, 0
}}
\only<6-6>{
{\color{blue}
	ret i32 0
}}
\only<7-7>{
{\color{blue}
\%a = shl i32 \%b, 2
}}
}}\\
{\color{green}optimierter Code}}

\end{column}%
\end{columns}

\end{frame}

\begin{frame}{Fazit / Ausblick}
\begin{itemize}
\item Es wurden mehrere Optimierungsalgorithmen erfolgreich auf einen gegebenen LLVM-Code angewendet. Auch bei natürlichem, aus C-Code über CLANG erzeugten LLVM-Code wurden erfolgreich bedeutende Mengen an Codezeilen entfernt.
\vspace{3mm}
\item Eine Weiterentwicklung des Programms wäre durch die Vielzahl an potentiellen weiteren Optimierungstechniken problemlos möglich. Priorität hätte hierbei die Anwendung von Schleifenoptimierungen.
\end{itemize}
\end{frame}

\begin{frame}{Danke}
	\begin{center}
		\begin{Large}Vielen Dank für Ihre Aufmerksamkeit.\end{Large}
	\end{center}
\end{frame}

% -------------------------------------------------------------------------------------------
% -------------------------------------------------------------------------------------------

\begin{frame}{Quellen}
	\begin{itemize}
		\item Aho, Alfred V. ; Lam, Monica S. ; Sethi, Ravi; Ullman, Jeffrey D.: Compilers: Principles, Techniques and Tools.
		\vspace{2mm}
		\item LLVM Language Reference Manual. http://llvm.org/docs/LangRef.html (Abruf 17.07.2012).
	\end{itemize}
\end{frame}

\end{document}
